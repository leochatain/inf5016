\documentclass{article}
\usepackage[brazil]{babel}
\usepackage[T1]{fontenc}
\usepackage[utf8]{inputenc}
\usepackage{graphicx}
\usepackage{times}
\usepackage{framed}
\usepackage{biblatex}
\usepackage{xcolor}
\usepackage{hyperref}

\colorlet{shadecolor}{orange!15}

\title{Laboratório 2}
\author{Leonardo Chatain}

\addbibresource{ex1.bib}

\begin{document}
\maketitle

\section{Tarefa}
Implementar o algoritmo de Dijkstra, que permite calcular o menor caminho entre um nodo e qualquer outro de um grafo. Como estrutura de heap, foi utilizada uma \emph{Árvore de Van Emde Boas}, logo a complexidade do Dijkstra é $O((m + n) \log\log n)$. Validar experimentalmente que o desempenho é $O((m + n) \log\log n)$. Determinar a constante de proporcionalidade $c$, tal que o tempo é $c*(m + n) \log\log n$.

\section{Solução}
Implementei o algoritmo de Dijkstra usando um heap binomial, como visto em aula. Heaps de Van Emde Boas realizam as operações \emph{insert} - no código: \emph{push(T val)} -, \emph{deletemin} - no código: \emph{pop()} -, e \emph{update} - no código: \emph{update(T val, T new)} com complexidade $ \log n $ (onde \emph{n} é o número de vértices do grafo).

Para o desenvolvimento foi utilizada a linguagem C++ e o compilador gcc versão 4.6. Para os testes foi utilizado o framework CppUnit.

\section{Ambiente de teste}

Os resultados foram obtidos utilizando-se um \emph{Intel core i7}, com um processador de $2.27$ GHz e $8$ GB de RAM.

Para os testes do algoritmo de Dijkstra foram utilizados grafos gerados randomicamente de tamanho $n=2^10,2^11,\ldots,2^14$. Todos os grafos utilizados eram completos, ou seja, o número de arestas $m=n ^ 2$

Cada teste foi repetido $10$ vezes e foi tomada a média dos tempos de execução.

\section{Resultados}

\subsection{Heap de Van Emde Boas}

A complexidade das operações \emph{insert}, \emph{deletemin} e \emph{update} é $ \log\log n $.

Para executar os testes, fizeram-se $ n $ repetidas operações ao heap de tamanho $ n $ (n inserts, n deletemins, n updates), o que ao todo tem uma complexidade de $ n \log\log n $

A Figura~\ref{fig3} mostra o tempo de execução da operação \emph{insert}, com complexidade $\log\log n$ executada $n$ vezes. A Figura~\ref{fig6} mostra o tempo de execução da operação \emph{insert}, dividido pela complexidade teórica $n * \log\log n$.

A Figura~\ref{fig3} mostra o tempo de execução da operação \emph{deletemin}, com complexidade $\log\log n$ executada $n$ vezes. A Figura~\ref{fig6} mostra o tempo de execução da operação \emph{insert}, dividido pela complexidade teórica $n * \log\log n$.

A Figura~\ref{fig3} mostra o tempo de execução da operação \emph{update}, com complexidade $\log\log n$ executada $n$ vezes. A Figura~\ref{fig6} mostra o tempo de execução da operação \emph{insert}, dividido pela complexidade teórica $n * \log\log n$.

\begin{figure}
  \centering
  \includegraphics[width=\textwidth,keepaspectratio]{plot_push.png}
  \caption{Tempo de execução $T_o$ da operação insert em um Heap de Van Emde Boas.}
  \label{fig3}
\end{figure}

\begin{figure}
  \centering
  \includegraphics[width=\textwidth,keepaspectratio]{plot_pop.png}
  \caption{Tempo de execução $T_o$ da operação deletemin em um Heap de Van Emde Boas.}
  \label{fig4}
\end{figure}

\begin{figure}
  \centering
  \includegraphics[width=\textwidth,keepaspectratio]{plot_update.png}
  \caption{Tempo de execução $T_o$ da operação update em um Heap de Van Emde Boas.}
  \label{fig5}
\end{figure}

\begin{figure}
  \centering
  \includegraphics[width=\textwidth,keepaspectratio]{plot_push_invert.png}
  \caption{Tempo de execução $T_o$ da operação insert em um Heap de Van Emde Boas, dividido pela complexidade teórica $\log\log n$}
  \label{fig6}
\end{figure}

\begin{figure}
  \centering
  \includegraphics[width=\textwidth,keepaspectratio]{plot_pop_invert.png}
  \caption{Tempo de execução $T_o$ da operação deletemin em um Heap de Van emde Boas, dividido pela complexidade teórica $\log\log n$}
  \label{fig7}
\end{figure}

\begin{figure}
  \centering
  \includegraphics[width=\textwidth,keepaspectratio]{plot_update_invert.png}
  \caption{Tempo de execução $T_o$ da operação update em um Heap de Van Emde Boas, dividido pela complexidade teórica $\log\log n$}
  \label{fig8}
\end{figure}

\subsection{Dijkstra}

Como mencionado anteriormente, todos os grafos utilizados são completos, portanto $m=n^2$ e a complexidade do algoritmo é $(n^2 + n) \log\log n$.

Para executar os testes, fizeram-se 10 queries aleatórias a grafos completos de tamanhos variados.

A Tabela~\ref{tab1} mostra o tempo de execução do algoritmo de Dijkstra para um grafo completo com $n=2^i * i$, $i=10,\ldots,15$ vértices e $n ^ 2$ arestas.


\begin{table}
  \centering
  \begin{tabular}{rrrrrrrrrrrrrr}
    \hline
    n =             & 512  & 1024 & 1536 & 2048 & 2560 & 3072        \\
    $T_o$ [$s$]     & 0.01 & 0.11 & 0.31 & 0.89 & 1.70 & 2.56     \\
    \hline
    n =             & 3584 & 4096  & 4608 & 5120  & 5632  & 6144  \\
    $T_o$ [$s$]     & 4.37 & 6.20  & 8.44 & 12.20 & 18.13 & 23.69 \\
    \hline
    n =             & 6656  & 7168  & 7680  & 8192  \\
    $T_o$ [$s$]     & 39.20 & 47.19 & 62.44 & 86.01 \\
    \hline
  \end{tabular}
  \caption{Tempo de execução $T_o$ do algoritmo de Dijkstra para um grafo completo com $n=512 * i$, $i=1,\ldots,16$ vértices e $n ^ 2$ arestas.}
  \label{tab1}
\end{table}

O gráfico do tempo de execução é plotado na Fig.~\ref{fig1}.

\begin{figure}
  \centering
  \includegraphics[width=\textwidth,keepaspectratio]{plot.png}
  \caption{Tempo de execução $T_o$ do algoritmo de Dijkstra para um grafo com $n=512 * i$, $i=1,\ldots,16$ vértices e $n ^ 2$ arestas.}
  \label{fig1}
\end{figure}

A Fig.~\ref{fig2} mostra o gráfico do tempo de execução normalizado $T_o/(n^2+n)\log\log n$. Esse gráfico é muito semelhante a uma reta constante (note que a amplitude vai de $1*10 ^ {-9}$ a $1.8 * 10 ^ {-9}$).

\begin{figure}
  \centering
  \includegraphics[width=\textwidth,keepaspectratio]{plot_invert.png}
  \caption{Tempo de execução normalizado $T_o/(n^2+n)\log\log n$ do algoritmo de Dijkstra para um grafo com $n=512 * i$, $i=1,\ldots,16$ vértices e $n ^ 2$ arestas.}
  \label{fig2}
\end{figure}

\section{Conclusão}

Podemos ver que o tempo de execução do algoritmo de Dijkstra segue a complexidade teórica.

Podemos também ver que o Heap de Van Emde Boas segue a complexidade esperada.

\printbibliography

\end{document}
% Local Variables:
% auto-fill-function: do-auto-fill
% TeX-PDF-mode: t
% fill-column: 110
% ispell-local-dictionary: "brasileiro"
% mode-name: "LaTeX"
% End:
