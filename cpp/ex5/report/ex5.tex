\documentclass{article}
\usepackage[brazil]{babel}
\usepackage[T1]{fontenc}
\usepackage[utf8]{inputenc}
\usepackage{graphicx}
\usepackage{times}
\usepackage{framed}
\usepackage{biblatex}
\usepackage{xcolor}
\usepackage{hyperref}

\title{Laboratório 5}
\author{Leonardo Chatain}

\begin{document}
\maketitle

\section{Tarefa}
Implementar três diferentes tabelas hash:

\begin{description}
 \item[Encadeamento] Nesta versão chaves que colidem são armazenadas na mesma posição na
tabela, em uma lista encadeada.

Para buscar uma chave nesta tabela é preciso percorrer a lista encadeada da posição
correspondente.

 \item[Endereçamento aberto] Nesta versão, chaves que colidem são armazenadas em outras
posições da própria tabela. Na versão implementada (\emph{linear probing}), caso haja uma colisão,
tenta-se inserir a chave em posições subsequentes da tabela.

 \item[Cuckoo hashing] Nesta versão existem duas funções hash. Uma chave está na tabela se ela
estiver em pelo menos uma das duas entradas correspondentes.

Para inserir a chave, caso haja uma colisão, ``chuta-se'' o elemento atual e insere-se o elemento
nessa posição. A seguir, tenta-se inserir o elemento chutado na sua outra posição (potencialmente
chutando outro elemento).

Casualmente haverá um ciclo na busca. Nesse caso, a tabela é reconstruida com novas entradas.

\end{description}

\section{Solução}

\subsection{Funções Hash}

As funções hash usadas são do seguinte tipo:

$$ H(x) = ((h_0 + h_1 * x + h_2 * x^2 + ... + h_n * x^n) mod P) mod M $$

Onde $n$ é o grau da função hash (pode ser escolhido), $M$ é o tamanho da tabela hash e P é um
primo ($P > M$).

Inicialmente usei como primo $2^{31} - 1$. Entretanto, com um primo muito maior que qualquer número
utilizado, o modulo P não acrescentava muita randomização ($ x mod P = x$ para $x < P$), e isso
causava loops muito frequentes na tabela cuckoo.

Entretanto, segundo o postulado de Bertrands, para todo número $n > 2$ existe um primo $p$ tal que
$n < p < 2 * n - 2$. Com isso podemos encontrar o menor primo maior que $M$ em $O(M)$ testes. Para
os testes, inicialmente quis usar o algoritmo de Miller-Rabin, mas como a busca por um primo só
seria feita na inicialização da tabela, optei por usar a solução simples $O(\sqrt{M})$.

No código, funções hash são calculadas pela classe \texttt{HashFunction}, e pode-se configurar o
grau da função, bem como a semente para randomização do polinômio.

\section{Ambiente de teste}

Os resultados foram obtidos utilizando-se um \emph{Intel core i5}, com um processador de $2.27$ GHz
e $4$ GB de RAM.

Para os testes do algoritmo de Dijkstra foram utilizados grafos gerados randomicamente de tamanho
$n=512,1024,1536,\ldots,8192$. Todos os grafos utilizados eram completos, ou seja, o número de
arestas $m=n ^ 2$

Cada teste foi repetido $10$ vezes e foi tomada a média dos tempos de execução.

\section{Resultados}

\end{document}
