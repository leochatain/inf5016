\documentclass{article}
\usepackage[brazil]{babel}
\usepackage[T1]{fontenc}
\usepackage[utf8]{inputenc}
\usepackage{graphicx}
\usepackage{times}
\usepackage{framed}

\title{Laboratório 3}
\author{Leonardo Chatain}

\begin{document}
\maketitle

\section{Tarefa}

Implementar o algoritmo de Ford-Fulkerson com a estratégia do ``caminho mais gordo'' (fattest path)
s-t.

Verificar que a complexidade observada é $ O((nlogn + m)m log C) $.

\section{Solução}

Implementei o algoritmo de Ford-Fulkerson usando:

\begin{description}
 \item[grafo] Um multimap (\texttt{map<int, map<int, int>}) para o \emph{grafo}. Isso permite o
acesso as arestas do grafo residual em tempo constante.

 \item[fattest-path] Um dijkstra-like algoritmo para encontrar o fattest path. A ideia é
sempre escolher a aresta com maior gargalo disponivel (usando um max-heap), em que o gargalo é o
min das arestas do caminho. É facil modificar o dijkstra para fazer isso.

 \item[priority-queue] usei uma \texttt{stl::priority\_queue} para o heap (o heap que eu tinha antes
era um hardcoded minheap, e a stl era mais facil e rapida de usar. Uma priority-queue tem
complexidades $O(\log n)$ para tudo.
\end{description}

\section{Ambiente de teste}

Os resultados foram obtidos utilizando-se um \emph{Intel core i5}, com um processador de $2.27$ GHz
e $4$ GB de RAM.

\section{Resultados}

\section{Conclusão}

\end{document}
% Local Variables:
% auto-fill-function: do-auto-fill
% TeX-PDF-mode: t
% fill-column: 110
% ispell-local-dictionary: "brasileiro"
% mode-name: "LaTeX"
% End:
