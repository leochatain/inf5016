\documentclass{article}
\usepackage[brazil]{babel}
\usepackage[T1]{fontenc}
\usepackage[utf8]{inputenc}
\usepackage{graphicx}
\usepackage{times}
\usepackage{framed}

\title{Laboratório 3}
\author{Leonardo Chatain}

\begin{document}
\maketitle

\section{Tarefa}

Implementar o algoritmo de Ford-Fulkerson com a estratégia do ``caminho mais gordo'' (fattest path)
s-t.

Verificar que a complexidade observada é $ O((nlogn + m)m log C) $.

\section{Solução}

Implementei o algoritmo de Ford-Fulkerson usando:

\begin{description}
 \item[grafo] Um multimap (\texttt{unordered\_map<int, unordered\_map<int, int>}) para o
\emph{grafo}. Isso permite o
acesso as arestas do grafo residual em tempo constante.

 \item[fattest-path] Um dijkstra-like algoritmo para encontrar o fattest path. A ideia é
sempre escolher a aresta com maior gargalo disponivel (usando um max-heap), em que o gargalo é o
min das arestas do caminho. É facil modificar o dijkstra para fazer isso.

 \item[priority-queue] usei uma \texttt{stl::priority\_queue} para o heap (o heap que eu tinha antes
era um hardcoded minheap, e a stl era mais facil e rapida de usar. Uma priority-queue tem
complexidades $O(\log n)$ para tudo.
\end{description}

\section{Ambiente de teste}

Os resultados foram obtidos utilizando-se um \emph{Intel core i5}, com um processador de $2.27$ GHz
e $4$ GB de RAM.

Cada experimento foi executado 10 vezes, a média aritmética foi tomada.

\section{Resultados}

O algoritmo foi rodado contra grafos gerados com a função \emph{mesh (1)}. Os resultados do
algoritmo são mostrados na Figura~\ref{fig:grafico}. Na Figura~\ref{fig:grafico_inv} podemos ver a
curva de tempo dividida pela complexidade esperada $ O((n\log n + m)m * \log C) $. Nota: quando
dividimos o tempo pelo tempo esperado, ignoramos o fator $\log C$.

Na Figura~\ref{fig:grafico_inv} vemos que a complexidade esperada nao define exatamente a
complexidade encontrada na pratica. Entretanto, o interessante é que parece que a complexidade
esperada é \emph{maior} que a complexidade vista na pratica (mesmo tendo ignorado o elemento $\log
C$ (o grafico parece algo como $1/x$).

Isso parece curioso, mas acredito que tenha a ver com a característica do grafo. O grafo é um
\emph{mesh}, com source e sink de lados opostos e arestas indo do source para o sink. Isso faz com
que o tamanho maximo de um caminho seja o numero de colunas no mesh (no nosso caso, os meshes são
todos quadrados, então estamos falando da raiz quadrada do número de vértices).

Se levarmos em consideração que o tamanho máximo de um caminho é aproximadamente a raiz do número de
vértices, então temos que a complexidade foi super-estimada e nossos resultados passam a fazer mais
sentido (especialmente com a ``cara'' $1/x$ desses resultados).

\begin{figure}
 \center
 \includegraphics[width=\textwidth,keepaspectratio]{fig1_unordered}
 \caption{Tempo por tamanho do grafo (grafo do tipo 1 : mesh)}
 \label{fig:grafico}
\end{figure}

\begin{figure}
 \center
 \includegraphics[width=\textwidth,keepaspectratio]{fig1_invert}
 \caption{Tempo de execuçaõo normalizado pela complexidade esperada.}
 \label{fig:grafico_inv}
\end{figure}


\section{Conclusão}

O algoritmo foi relativamente fácil de se programar, mas não pudemos reaproveitar as estruturas de
dados anteriores (como explicado anteriormente).

Pudemos observar na prática que os resultados não seguem exatamente a complexidade esperada, mas
atribuímos isso às características do grafo.

\end{document}
% Local Variables:
% auto-fill-function: do-auto-fill
% TeX-PDF-mode: t
% fill-column: 110
% ispell-local-dictionary: "brasileiro"
% mode-name: "LaTeX"
% End:
