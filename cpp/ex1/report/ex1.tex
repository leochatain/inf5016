\documentclass{article}
\usepackage[latin1]{inputenc}   % pacote para acentua??o
\usepackage{graphicx}           % pacote para importar figuras
\usepackage{times}              % pacote para usar fonte Adobe Times
\usepackage{framed}             % para exemplos e TODOs
\usepackage{biblatex}           % para refer?ncias bibliogr?ficas
\usepackage{xcolor}             % cores
\usepackage{hyperref}           % refer?ncias

\colorlet{shadecolor}{orange!15}

\title{Laboratorio 1}
\author{}{J. Rapidez}

\addbibresource{ex1.bib}

\begin{document}
\maketitle

\section{Tarefa}
Implementar o algoritmo de X, que permite calcular um Y em tempo $O(T(n))$. Validar experimentalmente que o
desempenho ? $O(T(n))$. Determinar a constante de proporcionalidade $c$, tal que o tempo ? $c \log n$.

\section{Solução}
Implementei o algoritmo de X usando uma estrutura da dados Y. Um
problema particular na solu??o foi fazer Z, que resolvemos fazendo W.

\begin{shaded}
  Não existem restrições para a forma da solução: a escolha de estruturas de dado, por exemplo, e
  livre. Qualquer linguagem de programação que possui uma implementação (compilador, interpretador) em
  software livre, disponível para um Ubuntu Linux admissível.
\end{shaded}

\section{Ambiente de teste}

Os resultados foram obtidos numa Cray X-MP, com $24$ processadores de
$200$ MHz, e $256$ GB de RAM. Testamos com dados gerados
randomicamente de tamanho $n=2,4,8,\ldots,2^{12}$. Cada teste foi
repetido $20$ vezes.

\section{Resultados}

\begin{shaded}
  Dicas para avaliar o tempo:
  \begin{itemize}
  \item Sempre se for poss?vel, n?o medir o tempo de uma execu??o de uma ?nica opera??o ou execu??o, mas de
    v?rias, e relatar o tempo m?dio. Uma diretiva simples ? n?o medir tempo menor que $1$s.
  \item Pode ser dif?cil comparar visualmente um gr?fico de medidas com o tempo te?rico esperado. Uma maneira
    para facilitar a compara??o ? dividir o tempo observado $T_o(n)$ pelo tempo esperado $T(n)$. Para
    $n\rightarrow\infty$ a curve deve se aproximar a uma constante.
  \item Uma outra abordagem com um modelo gen?rico~\parencite{Sedgewick/2011}: Supondo que o tempo de execu??o
    ? $T(n)=a\,n^c$, podemos avaliar o tempo para $n=n_0,2n_0,4n_0,\ldots$ e determinar $c$ como limite de
    $\log_2(T(2n)/T(n))$ e $a$ como limite de $T(n)/n^c$.
  \end{itemize}
\end{shaded}


A Tabela~\ref{tab1} mostra o tempo de execu??o do algoritmo de Dijkstra para um grafo com $n=2^i$,
$i=2,\ldots,12$ v?rtices ? $\approx 0.6n^2$ arestas. Aplicando o modelo gen?rico $a\,n^c$ obtemos $c=2$ e
$a=6\times 10^{-9}$. A complexidade pessimista te?rica ? $(n+m)\log n=(0.6n^2+n)\log n$.


\begin{table}
  \centering
  \begin{tabular}{rrrrrrrrrrrrrr}
    \hline
    n =             & 2     & 4      & 8       & 16      & 32       & 64        \\
    $T_o$ [$\mu s$] & 0.77  & 0.87   & 1.19    & 2.36    & 7.57     & 27.91     \\
    \hline
    n =             & 128   & 256    & 512     & 1024    & 2048     & 4096      \\
    $T_o$ [$\mu s$] & 92.52 & 336.13 & 1434.72 & 6024.10 & 24390.20 & 100000.00 \\
    \hline
  \end{tabular}
  \caption{Tempo de execu??o $T_o$ do algoritmo de Dijkstra para um grafo com
    $n=2^i$, $i=2,\ldots,12$ v?rtices ? $\approx 0.6n^2$ arestas.}
  \label{tab1}
\end{table}

\begin{shaded}
  Dicas:
  \begin{itemize}
  \item Alinhar colunas num?ricas sempre ? direita.
  \item Informar quantidades com um n?mero de d?gitos razo?vel e arrendondar corretamente.
  \end{itemize}
\end{shaded}

O tempo de execu??o dividido pelo tempo esperado ? mostrado na Fig.~\ref{fig1}.

\begin{figure}
  \centering
  \includegraphics{R}
  \caption{Tempo de execu??o normalizado $T_o/(0.6n^2+n)\log n$ do algoritmo de Dijkstra para um grafo com
    $n=2^i$, $i=2,\ldots,12$ v?rtices ? $\approx 0.6n^2$ arestas.}
  \label{fig1}
\end{figure}

\begin{shaded}
  Dicas para figuras:
  \begin{itemize}
  \item Rotular os eixos.
  \end{itemize}

\end{shaded}

\section{Conclus?o}

O algoritmo X comporta-se como esperado nos intervalos A e B e um melhor desempenho para valores C e D. Isso
pode ser explicado por Y.

\begin{shaded}
  Muito mais dicas: \textcite{Johnson/2002}.
\end{shaded}

\printbibliography

\end{document}
% Local Variables:
% auto-fill-function: do-auto-fill
% TeX-PDF-mode: t
% fill-column: 110
% ispell-local-dictionary: "brasileiro"
% mode-name: "LaTeX"
% End:
